% Created 2016-09-26 Mo 23:50
\documentclass[11pt]{article}
\usepackage[utf8]{inputenc}
\usepackage[T1]{fontenc}
\usepackage{fixltx2e}
\usepackage{graphicx}
\usepackage{longtable}
\usepackage{float}
\usepackage{wrapfig}
\usepackage{rotating}
\usepackage[normalem]{ulem}
\usepackage{amsmath}
\usepackage{textcomp}
\usepackage{marvosym}
\usepackage{wasysym}
\usepackage{amssymb}
\usepackage{hyperref}
\tolerance=1000
\usepackage{siunitx}
\usepackage{fontspec}
\newcommand{\estimates}{\overset{\scriptscriptstyle\wedge}{=}}
\author{Robin Heinemann}
\date{\today}
\title{Mathematischer Vorkurs}
\hypersetup{
  pdfkeywords={},
  pdfsubject={},
  pdfcreator={Emacs 25.1.1 (Org mode 8.2.10)}}
\begin{document}

\maketitle
\tableofcontents


\section{Messwert und Maßeinheit}
\label{sec-1}
Zu jeder phys. Größe gehören \uline{Messwert} und \uline{Maßeinheit}, d.h. Zahlewert $$\cdot$$ Einheit

\subsection{Beispiel}
\label{sec-1-1}
Geschw. $$v = \si{\kilo\meter\per\second}$$

\subsection{Bezeichungen}
\label{sec-1-2}
\begin{center}
\begin{tabular}{ll}
Abkürzung & Bedeutung\\
\hline
t & time\\
m & mass\\
v & velocity\\
a & acceleration\\
F & Force\\
E & Energy\\
T & Temperature\\
p & momentum\\
I & electric current\\
V & potential\\
\end{tabular}
\end{center}

Wenn das lateinische  Alphabet nicht ausreicht: griechische Buchstaben
$$\alpha, \beta, \gamma, \delta, \Delta, \Gamma, \epsilon, \zeta, \eta, \Theta, \kappa, \lambda, \mu, \nu, \Xi, \pi, \rho, \sigma, \tau, \phi, \chi, \psi, \omega, \Omega$$

\subsection{Maßeinheiten}
\label{sec-1-3}
Maßeinheiten werden über Maßstäbe definiert.

\subsubsection{Bespiel:}
\label{sec-1-3-1}
\SI{1}{\meter} = Strecke, die das Licht in $$\frac{1}{299792458}\si{\second}$$ zurücklegt.

\subsubsection{SI-Einheiten}
\label{sec-1-3-2}
Internationaler Standart (außer die bösen Amerikaner :D)

\begin{center}
\begin{tabular}{lll}
Größe & Einheit & Symbol\\
\hline
Länge & Meter & \si{\meter}\\
Zeit & Sekunden & \si{\second}\\
Masse & Kilogramm & \si{\kilogram}\\
elektrischer Strom & Ampere & \si{\ampere}\\
Temperatur & Kelvin & \si{\kelvin}\\
Lichstärke & Candela & \si{\candela}\\
ebener Winkel & Radiant & \si{\radian}\\
Raumwinkel & Steradiant & \si{\steradian}\\
Stoffmenge & Mol & \si{\mol}\\
\end{tabular}
\end{center}

\begin{enumerate}
\item Radiant
\label{sec-1-3-2-1}
Kreisumfang $$U = 2\pi r$$
Bogenmaß $$b = \phi r$$

Umrechung in Winkelgrad
$$\SI{2\pi}{\radian} \estimates \SI{360}{\degree}$$

$$\frac{Winkel in Radiant}{2\pi} = \frac{Winkel in Grad}{360}$$

\item Steradiant
\label{sec-1-3-2-2}
$$\Omega = \frac{A}{r^2}$$

\item Abgeleitete Einheiten
\label{sec-1-3-2-3}
\begin{center}
\begin{tabular}{llll}
Gröpe & Einheit & Symbol & Equivalent\\
\hline
Frequenz & Hertz & \si{\hertz} & \si{1\per\second}\\
Kraft & Newton & \si{\newton} & \si{\kilogram\meter\per\square\second}\\
Energie & Joule & \si{\joule} & \si{\newton\meter}\\
Leistung & Watt & \si{\watt} & \si{\joule\per\second}\\
Druck & Pascal & \si{\pascal} & \si{\newton\per\square\meter}\\
elektrischer Ladung & Coulomb & \si{\coulomb} & \si{\ampere\second}\\
elektrisches Potenzal & Volt & \si{\volt} & \si{\joule\per\coulomb}\\
elektrischer Wiederstand & Ohm & \si{\ohm} & \si{\volt\per\ampere}\\
Kapazität & Farad & \si{\farad} & \si{\coulomb\per\newton}\\
magn. Fluss & Weber & \si{\weber} & \si{\volt\per\second}\\
\end{tabular}
\end{center}

\item Prefix / Größenordungen
\label{sec-1-3-2-4}
\begin{center}
\begin{tabular}{lrl}
Prefix & $\log$\{10\} & Abkürzung\\
\hline
Dezi & -1 & d\\
Zenti & -2 & c\\
Milli & -3 & m\\
Mikro & -6 & $\mu$\\
Nano & -9 & n\\
Piko & -12 & p\\
Femto & -15 & f\\
Atto & -18 & a\\
Zepta & -21 & z\\
Yokto & -24 & y\\
Deka & 1 & D\\
Hekto & 2 & h\\
Kilo & 3 & k\\
Mega & 6 & M\\
Giga & 9 & G\\
Tera & 12 & T\\
Peta & 15 & P\\
Exa & 18 & E\\
Zetta & 21 & Z\\
Yotta & 24 & Y\\
\end{tabular}
\end{center}
\end{enumerate}

\section{Zeichen und Zahlen}
\label{sec-2}
\begin{center}
\begin{tabular}{ll}
Zeichen & Bedeutung\\
\hline
+ & plus\\
$\cdot$ & mal\\
= & gleich\\
< & ist kleiner als\\
> & ist größer als\\
$\angle$ & Windel zwischen\\
- & minus\\
$\neq$ & ungleich\\
$\le$ & kleiner gleich\\
$\ge$ & größer gleich\\
$\simeq$ & ungefähr gleich\\
\textpm{} & plus oder minus\\
$\perp$ & steht senkrecht auf\\
$\equiv$ & ist identisch gleich\\
$\ll$ & ist klein gegen\\
$\gg$ & ist groß gegen\\
$\infty$ & größer als jede Zahl\\
$\to$ $\infty$ & eine Größe wächst über alle Grenzen $\backslash$ Limes\\
$\sum$ & Summe\\
\end{tabular}
\end{center}

\subsection{Summenzeichn}
\label{sec-2-1}
\subsubsection{Beispiel}
\label{sec-2-1-1}
\begin{itemize}
\item $$\sum_{n=1}^3a_n=a_1 + a_2 + a_3$$
\item Summe der ersten $m$ natürlischen Zahlen
$$\sum_{n=1}^{m}n = 1 + 2 + \ldots + (m -1) + m = \frac{m (m + 1)}{2}$$
\item Summe der ersten $m$ Quadrate der natürlichen Zahlen
$$\sum_{n=1}^m n^2 = 1 + 4 + \ldots + (m-1)^2 + m^2 = \frac{m(m+1)(2m+1)}{6}$$
\item Summe der ersten $m$ Potenzen einer Zahl ($q \neq 1$)
$$\sum_{n=0}^m q^n = 1+q+\dots+q^{m-1}+q^m = \frac{1 - q^{m + 1}}{1-q}$$
sog. \emph{geometrische Summe}
\begin{itemize}
\item Beweis
$$s_m = 1 + \ldots + q^m$$
$$q s_m = q + \ldots + q^{m+1}$$
$$s_m - q s_m = s_m(1-q) = 1-q^{m+1}$$
\end{itemize}
\end{itemize}

\subsubsection{Rechenregeln}
\label{sec-2-1-2}
\begin{itemize}
\item $$\sum_{k=m}^n a_k = \sum_{j=m}^n a_j$$
\item $$c\sum_{k=m}^n a_k = \sum_{k=m}^n c a_k$$
\item $$\sum_{k=m}^n a_k \pm \sum_{j=m}{n} b_k = \sum_{k=m}^n (a_k \pm b_k)$$
\item $$\sum_{k=m}^n a_k + \sum_{k=n+1}^p a_k = \sum_{k=m}^{p} a_k$$
\item $$\sum_{k=m}^n a_k = \sum_{k=m+p}^{n+p} a_{k-p} = \sum_{k=m-p}^{n-p} a_{k+p}$$
\item $$(\sum_{i=1}^n a_i)(\sum_{j=1}^m b_j) = \sum_{i=1}^n \sum_{j=1}^m a_i b_j = \sum_{j=1}^m \sum_{i=1}^n a_i b_j$$
      falls $$n=m$$ $$\sum_{i,j=1}^n a_i b_j$$
\end{itemize}

\subsection{Produktzeichen}
\label{sec-2-2}
\subsubsection{Beispiel}
\label{sec-2-2-1}
$$\prod_{n=1}^3 a_n = a_1 a_2 a_3$$

\subsection{Fakultätszeichen}
\label{sec-2-3}
$$m! = 1 \cdot 2 \cdot \ldots \cdot (m-1) \cdot m = \prod_{n=1}^m n$$

\section{misc}
\label{sec-3}
mathe für physiker vs. analysis
klasuren gebündelt
auslandssemester
% Emacs 25.1.1 (Org mode 8.2.10)
\end{document}