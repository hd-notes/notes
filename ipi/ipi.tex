% Created 2016-10-25 Di 14:54
\documentclass[11pt]{article}
\usepackage[utf8]{inputenc}
\usepackage[T1]{fontenc}
\usepackage{fixltx2e}
\usepackage{graphicx}
\usepackage{longtable}
\usepackage{float}
\usepackage{wrapfig}
\usepackage{rotating}
\usepackage[normalem]{ulem}
\usepackage{amsmath}
\usepackage{textcomp}
\usepackage{marvosym}
\usepackage{wasysym}
\usepackage{amssymb}
\usepackage{hyperref}
\tolerance=1000
\usepackage{siunitx}%
\usepackage{fontspec}%
\sisetup{load-configurations = abbrevations}%
\newcommand{\estimates}{\overset{\scriptscriptstyle\wedge}{=}}%
\usepackage{mathtools}%
\DeclarePairedDelimiter\abs{\lvert}{\rvert}%
\DeclarePairedDelimiter\norm{\lVert}{\rVert}%
\DeclareMathOperator{\Exists}{\exists}%
\DeclareMathOperator{\Forall}{\forall}%
\def\colvec#1{\left(\vcenter{\halign{\hfil$##$\hfil\cr \colvecA#1;;}}\right)}
\def\colvecA#1;{\if;#1;\else #1\cr \expandafter \colvecA \fi}
\usepackage{minted}
\usepackage{makecell}
\usemintedstyle{perldoc}
\author{Robin Heinemann}
\date{\today}
\title{Einführung in die Anwendungsorientierte Informatik (Köthe)}
\hypersetup{
  pdfkeywords={},
  pdfsubject={},
  pdfcreator={Emacs 25.1.1 (Org mode 8.2.10)}}
\begin{document}

\maketitle
\tableofcontents


\section{Klausur 09.02.2016}
\label{sec-1}

\section{Was ist Informatik?}
\label{sec-2}
"Kunst" Aufgaben mit Computerprogrammen zu lösen.
\subsection{Teilgebiete}
\label{sec-2-1}
\subsubsection{theoretische Informatik (\textbf{ITH})}
\label{sec-2-1-1}
\begin{itemize}
\item Berechenbarkeit: Welche Probleme kann man mit Informatik lösen und welche prinzipiell nicht?
\item Komplexität: Welche Probleme kann man effizient lösen?
\item Korrektheit: Wie beweist man, dass das Ergebnis richtig ist? \\
      Echtzeit: Dass das richtige Ergebnis rechtzeitig vorliegt.
\item verteilte Systeme: Wie sichert man, dass verteilte Systeme korrekt kommunizieren?
\end{itemize}
\subsubsection{technische Informatik (\textbf{ITE})}
\label{sec-2-1-2}
\begin{itemize}
\item Auf welcher Hardware kann man Programme ausführen, wie baut man dies Hardware?
\item CPU, GPU, RAM, HD, Display, Printer, Networks
\end{itemize}
\subsubsection{praktische Informatik}
\label{sec-2-1-3}
\begin{itemize}
\item Wie entwickelt man Software?
\item Programmiersprachen und Compiler: Wie kommuniziert der Programmierer mit der Hardware?\hfill \textbf{IPI}, \textbf{IPK}
\item Algorithmen und Datenstrukturen: Wie baut man komplexe Programme aus einfachen Grundbausteinen?\hfill \textbf{IAL}
\item Softwaretechnik: Wie organisiert man sehr große Projekte?\hfill \textbf{ISW}
\item Kernanwendung der Informatik: Betriebsysteme, Netzwerke, Parallelisierung\hfill \textbf{IBN}
\begin{itemize}
\item Datenbanksysteme\hfill \textbf{IDB1}
\item Graphik, Graphische Benutzerschnittstellen\hfill \textbf{ICG1}
\item Bild- und Datenanalyse
\item maschinelles Lernen
\item künstliche Intelligenz
\end{itemize}
\end{itemize}
\subsubsection{angewante Informatik}
\label{sec-2-1-4}
\begin{itemize}
\item Wie löst man Probleme aus einem anderem Gebiet mit Programmen?
\item Informationstechnik
\begin{itemize}
\item Buchhandlung, e-commerce, Logistik
\end{itemize}
\item Web programming
\item scientific computing für Physik, Biologie
\item Medizininformatik
\begin{itemize}
\item bildgebende Verfahren
\item digitale Patientenakte
\end{itemize}
\item computer linguistik
\begin{itemize}
\item Sprachverstehen, automatische Übersetzung
\end{itemize}
\item Unterhaltung: Spiele, special effect im Film
\end{itemize}
\section{Wie unterscheidet sich Informatik von anderen Disziplinen?}
\label{sec-3}
\subsection{Mathematik}
\label{sec-3-1}
Am Beispiel der Definition $a \leq b: \exists c \geq 0: a + c = b$
Informatik: Lösungsverfahren: $a - b \leq 0$, das kann man leicht ausrechen, wenn man subtrahieren und mit $0$ vergleichen kann.
Quadratwurzel: $y = \sqrt{x} \Leftrightarrow y \geq 0 \wedge y^2 = x (\Rightarrow x > 0)$
Informatik: Algorithmus aus der Antike: $y = \frac{x}{y}$
iteratives Verfahren: Initial Guess $y^{(0)} = 1$
schrittweise Verbesserung $y^{(t+1)} = \frac{y^{(t)} + \frac{x}{y^{(t)}}}{2}$
\section{Informatik}
\label{sec-4}
Lösugswege, genauer Algorithmen
\subsection{Algorithmus}
\label{sec-4-1}
\textbf{schematische} Vorgehensweise mit der jedes Problem einer bestimmten \textbf{Klasse} mit \textbf{endliche} vielen \textbf{elementaren} Schritten / Operationen gelöst werden kann
\begin{itemize}
\item schematisch: man kann den Algorithmus ausführen, ohne ihn zu verstehen ($\Rightarrow$ Computer)
\item alle Probleme einer Klasse: zum Beispiel: die Wurzel aus jeder beliebigen nicht-negativen Zahl, und nicht nur $\sqrt{11}$
\item endliche viele Schritte: man kommt nach endlicher Zeit zur Lösung
\item elementare Schrite / Operationen: führen die Lösung auf Operationen oder Teilprobleme zurück, die wir schon gelöst haben
\end{itemize}
\subsection{Daten}
\label{sec-4-2}
Daten sind Symbole,
\begin{itemize}
\item die Entitäten und Eigenschaften der realen Welt im Computer representieren.
\item die interne Zwischenergebnisse eines Algorithmus aufbewahren
\end{itemize}
$\Rightarrow$ Algorithmen transformieren nach bestimmten Regeln die Eingangsdaten (gegebene Symbole) in Ausgangsdaten (Symbole für das Ergebniss).
Die Bedeutung / Interpretation der Symbole ist dem Algorithmus egal $\estimates$ "schematisch"
\subsubsection{Beispiele für Symbole}
\label{sec-4-2-1}
\begin{itemize}
\item Zahlen
\item Buchstaben
\item Icons
\item Verkehrszeichen
\end{itemize}
aber: heutige Computer verstehen nur Binärzahlen $\Rightarrow$ alles andere muss man übersetzen
Eingansdaten: "Ereignisse":
\begin{itemize}
\item Symbol von Festplatte lesen oder per Netzwerk empfangen
\item Benutzerinteraktion (Taste, Maus, \ldots{})
\item Sensor übermittelt Meßergebnis, Stoppuhr läuft ab
\end{itemize}
Ausgangsdaten: "Aktionen":
\begin{itemize}
\item Symbole auf Festplatte schreiben, per Netzwerk senden
\item Benutzeranzeige (Display, Drucker, Ton)
\item Stoppuhr starten
\item Roboteraktion ausführen (zum Beispiel Bremsassistent)
\end{itemize}
Interne Daten:
\begin{itemize}
\item Symbole im Hauptspeicher oder auf Festplatte
\item Stoppuhr starten / Timeout
\end{itemize}
\subsection{Einfachster Computer}
\label{sec-4-3}
endliche Automaten (endliche Zustandsautomaten)
\begin{itemize}
\item befinden sich zu jedem Zeitpunkt in einem bestimmten Zustand aus einer vordefinierten endlichen Zustandsmenge
\item äußere Ereignisse können Zustandsänderungen bewirken und Aktionen auslösen
\end{itemize}
\subsubsection{{\bfseries\sffamily TODO} Graphische Darstellung}
\label{sec-4-3-1}
graphische Darstellung: Zustände = Kreise, Zustandsübergänge: Pfeile
\subsubsection{{\bfseries\sffamily TODO} Darstellung durch Übergangstabellen}
\label{sec-4-3-2}
Zeilen: Zustände, Spalten: Ereignisse, Felder: Aktion und Folgezustände
\begin{center}
\begin{tabular}{llll}
Zustände $\backslash$ Ereignisse & Knopf drücken & Timeout & Timeout(Variante)\\
\hline
aus & $\Rightarrow$\{halb\} \\ \{4 LEDs an\} & \% & ($\Rightarrow$\{aus\},\{nichts\})\\
halb & ($\Rightarrow$\{voll\},\{8 LEDs an\}) & \% & ($\Rightarrow$\{aus\},\{nichts\})\\
voll & ($\Rightarrow$\{blinken an\},\{Timer starten\}) & \% & ($\Rightarrow$\{aus\},\{nichts\})\\
blinken an & ($\Rightarrow$\{aus\},\{Alle LEDs aus, Timer stoppen\}) & ($\Rightarrow$\{blinken aus\},\{alle LEDs aus, Timer starten\}) & ($\Rightarrow$\{blinken aus\},\{8 LEDs aus\})\\
blinken aus & ($\Rightarrow$\{aus\},\{Alle LEDs aus, Timer stoppen\}) & ($\Rightarrow$\{blinken an\},\{alle LEDs an, Timer starten\}) & ($\Rightarrow$\{blinken an\},\{8 LEDs an\})\\
\end{tabular}
\end{center}

Variante: Timer läuft immer (Signal alle 0.3s) $\Rightarrow$ Timout ignorieren im Zustand "aus", "halb", "voll"
\subsubsection{Beispiel 2:}
\label{sec-4-3-3}
\begin{align}
&1~0~1~1~0~1~0 &= 2 + 8 + 16 + 74 &= 90_{\text{dez}} \\
+&0~1~1~1~0~0~1 &= 1 + 8 + 16 + 32 &= 57_{\text{dez}} \\
\hline
1~&0~0~1~0~0~1~1 &= 1 + 2 + 16 + 128 &= 147_{\text{dez}}\checkmark
\end{align}
% Emacs 25.1.1 (Org mode 8.2.10)
\end{document}