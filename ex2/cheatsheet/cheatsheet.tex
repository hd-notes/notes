\documentclass[9pt, landscape,a4paper]{extarticle}
\usepackage{tikz}
\usepackage{multicol}
\usepackage[top=1cm,bottom=1cm,left=1cm,right=1cm]{geometry}
\usepackage[ngerman]{babel}
\usepackage{microtype}
\usepackage{amsfonts}
\usepackage{amssymb}
\usepackage{mathtools}
\usepackage{siunitx}
%\usepackage{mathspec}
\usepackage{nath}
%\usepackage{unicode-math}
\usepackage{stmaryrd}
\usepackage{stackengine}
%\usepackage{polyglossia}
\usepackage{newunicodechar}
\newunicodechar{α}{\alpha}
\newunicodechar{Α}{\alpha}
\newunicodechar{β}{\beta}
\newunicodechar{Β}{\beta}
\newunicodechar{γ}{\gamma}
\newunicodechar{δ}{\delta}
\newunicodechar{ε}{\varepsilon}
\newunicodechar{Ε}{\varepsilon}
\newunicodechar{ζ}{\zeta}
\newunicodechar{η}{\eta}
\newunicodechar{θ}{\theta}
\newunicodechar{ι}{\iota}
\newunicodechar{κ}{\kappa}
\newunicodechar{λ}{\lambda}
\newunicodechar{μ}{\mu}
\newunicodechar{Μ}{M}
\newunicodechar{ν}{\nu}
\newunicodechar{ξ}{\xi}
\newunicodechar{ο}{\omicron}
\newunicodechar{π}{\pi}
\newunicodechar{Π}{\Pi}
\newunicodechar{ρ}{\rho}
\newunicodechar{Ρ}{P}
\newunicodechar{σ}{\sigma}
\newunicodechar{τ}{\tau}
\newunicodechar{Τ}{T}
\newunicodechar{υ}{\upsilon}
\newunicodechar{φ}{\varphi}
\newunicodechar{χ}{\chi}
\newunicodechar{ψ}{\psi}
\newunicodechar{ω}{\omega}
\newunicodechar{∀}{\forall}
\newunicodechar{×}{\times}
\newunicodechar{Γ}{\Gamma}
\newunicodechar{Δ}{\Delta}
\newunicodechar{∃}{\exists}
\newunicodechar{ℤ}{\mathbb{Z}}
\newunicodechar{∧}{\wedge}
\newunicodechar{Θ}{\Theta}
\newunicodechar{⇒}{\implies}
\newunicodechar{∩}{\cap}
\newunicodechar{Λ}{\Lambda}
\newunicodechar{∫}{\int}
\newunicodechar{ℕ}{\mathbb{N}}
\newunicodechar{Ξ}{\Xi}
\newunicodechar{∇}{\nabla}
\newunicodechar{Π}{\Pi}
\newunicodechar{ℝ}{\mathbb{R}}
\newunicodechar{Σ}{\Sigma}
\newunicodechar{⇔}{\iff}
\newunicodechar{Υ}{\Upsilon}
\newunicodechar{Φ}{\Phi}
\newunicodechar{ℂ}{\mathbb{C}}
\newunicodechar{Ψ}{\Psi}
\newunicodechar{Ω}{\Omega}
\newunicodechar{ϑ}{\vartheta}
\newunicodechar{∞}{\infty}
\newunicodechar{∈}{\in}
\newunicodechar{⊂}{\subset}
\newunicodechar{ϰ}{\varkappa}
\newunicodechar{ϕ}{\phi}
\newunicodechar{∨}{\vee}
\newunicodechar{∮}{\oint}
\newunicodechar{↦}{\mapsto}
\newunicodechar{ℚ}{\mathbb{Q}}
\newunicodechar{⊆}{\subseteq}
\newunicodechar{⊊}{\subsetneq}
\newunicodechar{∪}{\cup}
\newunicodechar{·}{\cdot}
%\setdefaultlanguage[spelling=new, babelshorthands=true]{german}
\makeatletter
\let\mathop\o@mathop
\makeatother

%\definecolor{myblue}{cmyk}{1,.72,0,.38}
\definecolor{myblue}{cmyk}{1,1,1, 1}
\def\firstcircle{(0,0) circle (1.5cm)}
\def\secondcircle{(0:2cm) circle (1.5cm)}

\everymath\expandafter{\the\everymath \color{myblue}}
\everydisplay\expandafter{\the\everydisplay \color{myblue}}

\renewcommand{\baselinestretch}{.8}
\pagestyle{empty}
\setlength{\mathindent}{0pt}

\makeatletter
\renewcommand{\section}{\@startsection{section}{1}{0mm}%
                                {.2ex}%
                                {.2ex}%x
                                {\sffamily\small\bfseries}}
\renewcommand{\subsection}{\@startsection{subsection}{1}{0mm}%
                                {.2ex}%
                                {.2ex}%x
                                {\sffamily\bfseries}}
\renewcommand{\subsubsection}{\@startsection{subsubsection}{1}{0mm}%
                                {.2ex}%
                                {.2ex}%x
                                {\sffamily\small\bfseries}}



\def\multi@column@out{%
   \ifnum\outputpenalty <-\@M
   \speci@ls \else
   \ifvoid\colbreak@box\else
     \mult@info\@ne{Re-adding forced
               break(s) for splitting}%
     \setbox\@cclv\vbox{%
        \unvbox\colbreak@box
        \penalty-\@Mv\unvbox\@cclv}%
   \fi
   \splittopskip\topskip
   \splitmaxdepth\maxdepth
   \dimen@\@colroom
   \divide\skip\footins\col@number
   \ifvoid\footins \else
      \leave@mult@footins
   \fi
   \let\ifshr@kingsaved\ifshr@king
   \ifvbox \@kludgeins
     \advance \dimen@ -\ht\@kludgeins
     \ifdim \wd\@kludgeins>\z@
        \shr@nkingtrue
     \fi
   \fi
   \process@cols\mult@gfirstbox{%
%%%%% START CHANGE
\ifnum\count@=\numexpr\mult@rightbox+2\relax
          \setbox\count@\vsplit\@cclv to \dimexpr \dimen@-1cm\relax
% \setbox\count@\vbox to \dimen@{\vbox to 1cm{\header}\unvbox\count@\vss}%
\else
      \setbox\count@\vsplit\@cclv to \dimen@
\fi
%%%%% END CHANGE
            \set@keptmarks
            \setbox\count@
                 \vbox to\dimen@
                  {\unvbox\count@
                   \remove@discardable@items
                   \ifshr@nking\vfill\fi}%
           }%
   \setbox\mult@rightbox
       \vsplit\@cclv to\dimen@
   \set@keptmarks
   \setbox\mult@rightbox\vbox to\dimen@
          {\unvbox\mult@rightbox
           \remove@discardable@items
           \ifshr@nking\vfill\fi}%
   \let\ifshr@king\ifshr@kingsaved
   \ifvoid\@cclv \else
       \unvbox\@cclv
       \ifnum\outputpenalty=\@M
       \else
          \penalty\outputpenalty
       \fi
       \ifvoid\footins\else
         \PackageWarning{multicol}%
          {I moved some lines to
           the next page.\MessageBreak
           Footnotes on page
           \thepage\space might be wrong}%
       \fi
       \ifnum \c@tracingmulticols>\thr@@
                    \hrule\allowbreak \fi
   \fi
   \ifx\@empty\kept@firstmark
      \let\firstmark\kept@topmark
      \let\botmark\kept@topmark
   \else
      \let\firstmark\kept@firstmark
      \let\botmark\kept@botmark
   \fi
   \let\topmark\kept@topmark
   \mult@info\tw@
        {Use kept top mark:\MessageBreak
          \meaning\kept@topmark
         \MessageBreak
         Use kept first mark:\MessageBreak
          \meaning\kept@firstmark
        \MessageBreak
         Use kept bot mark:\MessageBreak
          \meaning\kept@botmark
        \MessageBreak
         Produce first mark:\MessageBreak
          \meaning\firstmark
        \MessageBreak
        Produce bot mark:\MessageBreak
          \meaning\botmark
         \@gobbletwo}%
   \setbox\@cclv\vbox{\unvbox\partial@page
                      \page@sofar}%
   \@makecol\@outputpage
     \global\let\kept@topmark\botmark
     \global\let\kept@firstmark\@empty
     \global\let\kept@botmark\@empty
     \mult@info\tw@
        {(Re)Init top mark:\MessageBreak
         \meaning\kept@topmark
         \@gobbletwo}%
   \global\@colroom\@colht
   \global \@mparbottom \z@
   \process@deferreds
   \@whilesw\if@fcolmade\fi{\@outputpage
      \global\@colroom\@colht
      \process@deferreds}%
   \mult@info\@ne
     {Colroom:\MessageBreak
      \the\@colht\space
              after float space removed
              = \the\@colroom \@gobble}%
    \set@mult@vsize \global
  \fi}

\setlength{\parindent}{0pt}
\newcommand\ubar[1]{\stackunder[1.2pt]{\(#1\)}{\rule{1.25ex}{.08ex}}}

\expandafter\def\expandafter\normalsize\expandafter{%
    \normalsize
    \setlength\abovedisplayskip{-100pt}
    \setlength\belowdisplayskip{-100pt}
    \setlength\abovedisplayshortskip{-100pt}
    \setlength\belowdisplayshortskip{-100pt}
%    \setlength\beloweqnsskip{-100pt}
%    \setlength\displaybaselineskip{-100pt}
%    \setlength\displaylineskip{-100pt}
}

\renewcommand\v[1]{\vec{#1}}
\renewcommand\d{\mathrm{d}}
\renewcommand{\vec}[1]{\mathbf{#1}}
\newcommand*\abs[1]{\lvert#1\rvert}
\newcommand{\dd}[2]{\frac{\d #1}{\d #2}}
\newcommand{\pp}[2]{\frac{\partial #1}{\partial #2}}
\newcommand{\const}{\ensuremath{\text{ const.}}}%

\raggedbottom
\begin{document}
\small
\begin{multicols*}{4}
\raggedcolumns
\section{Elektrostatik}
Coulombsches Gesetz:
$\displayed{\v F_C = \frac{1}{4πε_0} \frac{q_1 q_2}{r_{12}^2} \hat r_{12}}$ \\
Elektrisches Feld: \\ $\v E(\v r) := \frac{\v F_C(\v r)}{q}, \v F(\v r) = q \v E(\v r)$ \\
Elekrische Feldstärke:
$\displayed{\v E(\v r) = \frac{1}{4πε_0} \frac{Q}{r^2} \hat r}$ \\
Superposition:
$\displayed{\v E(\v R) = \frac{1}{4π ε_0} ∫ \frac{\v R - \v r}{\abs{\v R - \v r}^3} ρ(\v r) \d V}$ \\
mit $Q = ∫ ρ(\v r) \d^3 r$ \\
diskrete Landungsverteilung: \\
$\displayed{\v E = \frac{1}{4 π ε_0} \sum \frac{q_i}{r_i^2} \hat r_i}$ \\
Elektrischer Fluss $ϕ_E := ∫ \v E \d \v A$: \\
$Q_{\text{innen}} = 0 \to ϕ_E = 0$ \\
$Q_{\text{innen}} > 0 \to ϕ_E > 0$ (Quelle) \\
$Q_{\text{innen}} < 0 \to ϕ_E < 0$ (Senke) \\
Gaußsches Gesetz:
$\displayed{∮ \v E \d \v A = \frac{Q_{\text{innen}}}{ε_0}}$ \\
Feld einer Linienladung: $E(r) = \frac{λ}{2π ε_0 r}$ \\
Feld einer Flächenladung: $E(r) = \frac{Γ}{2 ε_0}$ \\
Innerhalb von Leitern (auch in Hohlräumen): $\v E = 0, Q = 0$ \\
Gaußscher Satz:
$\displayed{∮_A \v E \d \v A = ∫_V \v ∇ · \v E \d V}$ \\
Gaußsches Gesetz (differentielle Form): $\v ∇ \v E = \frac{ρ}{ε_0}$ \\
Potentielle Energie: $E_{pot}(\v r) = - ∫_{∞}^{r} \v F_C \d \v s$ \\
Coulombpotential (Punktladung): $E_{pot}(\v r) = \frac{Q q}{4 π ε_0 r}$ \\
Zirkulationsgesetz: $∮ \v E \d \v s = 0$ \\
Elektrisches Potential: \\
$\displayed{φ(\v r) = \frac{E_{pot}(\v r)}{q} = -∫_{∞}^{\v r} \v E \d \v s}$ \\
$\v E(\v r) = -\v ∇ φ(\v r)$ \\
Allgemeine Ladungsverteilung: \\
$\displayed{φ(\v R) = \frac{1}{4 π ε_0} ∫ \frac{ρ(\v r)}{\abs{\v R - \v r}} \d V}$ \\
Elektrische Spannug: \\
$U_{12} = Δ φ_{12} = φ_2 - φ_1 = - ∫_1^2 \v E \d \v s$ \\
Poisson Gleichung: $Δ φ = - \frac{ρ}{ε_0}$ \\
Dipolmoment: $\v p = q · \v d$ \\
Dipol: $\displayed{φ(\v r) = \frac{1}{4π ε_0}(\frac{q}{\abs{\v r - \v d / 2}} + \frac{-q}{\abs{\v r + \v d / 2}})}$ \\
$r \gg d$: $\displayed{φ(\v r) = \frac{\v p · \hat r}{4 π ε_0 r^2}}$ \\
Feld für $r \gg d$: $\displayed{\v E(\v r) = \frac{3(\v p · \v r) \v r - r^2 \v p}{r^5}}$ \\
Homogenes Feld: $\v M = \v D × \v F = \v p × \v E$ \\
$E_{pot} = - \v p · \v E$ \\
Kugelkonduktoren: \\
$\displayed{Δ φ = - ∫_∞^R \v E \d \v r = \frac{Q}{4π ε_0 R}} \to Q = 4 π ε_0 R U$ \\
Kapazität: $C := \frac{Q}{U}$ \\
Plattenkondensator: $E = \frac{σ}{ε_0} = \frac{Q}{A ε_0}$ \\
$C = \frac{Q}{U} = \frac{ε_0 A}{d}$ \\
Kugelkondensator: $C= 4π ε_0 {\frac{R_{2} R_{1}}{R_{2} - R_{1}}}$ \\
Parallelschaltung: $C_{\text{ges}} = \sum_{i = 1}^n C_i$ \\
Reihenschaltung: $\frac{1}{C_{\text{ges}}} = \sum_{i = 1}^n \frac{1}{C_i}$ \\
Energie gepeichert im Kondensator: \\
$E_C = \frac{1}{2} C U^2$ \\
Plattenkondensator: $E_C = \frac{1}{2} ε_0 V E^2$ \\
Energiedichte: $ω_C = \frac{1}{2} ε_0 E^2$ \\
Permittivität $ε_r$: $C_{\text{Diel}} = ε_r C_{\text{Vakuum}} = ε_r C_0$ \\
$\displayed{E_{\text{Diel}} = \frac{E_{\text{Vak}}}{ε_r}}$ \\
Polarisation: $\v P = \frac{1}{V} \sum_{i = 1}^{n} \v p_i$ \\
$\v P = χ ε_0 \v E_{Diel}$ \\
elektrische Suszeptibilität $χ = ε_r - 1$ \\
Dielektrische Verschiebung: \\
$\v D = ε_0 \v E_{\text{Diel}} + \v P = ε_r ε_0 \v E_{\text{Diel}} = ε_0 \v E_{\text{Vak}}$ \\
1. Maxwellsche Gleichungen in Materie:
$\displayed{∮ \v D \d \v A = Q_{\text{frei}}}$ \\
$\displayed{∮ \v E \d \v A = \frac{Q_{\text{frei}}}{ε_r ε_0}}$ \\
Elektrische Feldenergie: $W_e = \frac{Q^2}{2 ε_r C_0}$ \\
Energiedichte: $ω_e = \frac{1}{2} ε_r ε_0 \v E^2 = \frac{1}{2} \v E \v D$ \\
\section{Elektrische Gleichströme}
Elektrischer Strom: $I := \dd{Q}{t}$ \\
Elektrische Stromdichte: $\abs{\v j} = \frac{I}{A} = \frac{\d Q}{A \d t}$ \\
Technische Stromrichtung: Flußrichtung der positiven Ladungsträger! \\
Ladungsfluss: $U = φ_b - φ_a = E Δl$ \\
Differentieller Widerstand: $r = \dd{U}{I}$ \\
Differentielle Leitfähigkeit: $S = \dd{I}{U}$ \\
Ohmscher Leiter $\to r = \const$ \\
Ohmsches Gesetz: $U = R I$ \\
$\v j = σ \v E = n q_e \v v_D$ \\
$I = \v j A$ \\
Spezifische Leitfähigkeit: $σ = \frac{l}{RA} = S \frac{l}{A}$ \\
Spezifische Widerstand: $ρ = \frac{1}{ρ} = R \frac{A}{l}$ \\
ohmsche Leiter: $\v v_D = μ \v E$ \\
Elektronenbeweglichkeit $μ = \frac{q τ}{m}$ \\
Mittlere Zeit zwischen zwei Wechselwirkungen: $τ$ \\
Elektrische Leistung: $P = U · I$ \\
Ohmscher Leiter: $F = R I^2, F = \frac{U^2}{R}$ \\
Kirchhoffsche Regeln: \\
Knotenregel: An jedem Konten gilt: $\sum I_k = 0$ \\
Maschenregel: Für jede Masche gilt: $\sum U_k = 0$ \\
Reihenschaltung: $R = \sum_{i = 1}^n R_i$ \\
Parallelschaltung: $\frac{1}{R} = \sum_{i = 1}^n \frac{1}{R_i}$ \\
Spannungsquelle ($R_i$: Innewiderstand): \\
$U_{kl} = U_0 - I R_i ⇒ R_i \to 0$ \\
Stromquelle: $R_i \to ∞$
\section{Magnetostatik}
$H$: magnetische Erregung \\
$B$: Magnetfeld und magnetische Flussdichte \\
Magnetischer Kraftfluss: $ϕ_m = ∫ \v B \d \v A$ \\
Lorentzkraft: $\v F = q(\v E + \v v × \v B)$ \\
Leiter: $\displayed{B = \frac{μ_0 I}{2 π r}}$ \\
$\v F = I (\v l × \v B)$ \\
Zyklotronfrequenz: $ω = \frac{q}{w} B$ \\
Leiterschleife: Magnetisches Moment: \\
$\v μ = I \v A = I A \v n$ \\
Drehmoment auf einen magnetischen Dipol: $\v M = \v μ × \v B$ \\
Hallspannung: \\
$U_H = \frac{I B}{nqd} = R_H \frac{I B}{d} $ \\
Leiter vekoriell: $\displayed{\v B(\v r) = \frac{μ_0 I}{2 π r}(\hat l × \hat r)}$ \\
Quellfreiheit: $∮_A \v B \d \v A = 0$ \\
Ampersches Durchflutungsgesetz: \\
$∮ \v B \d \v s = μ_0 \sum_k I_k = μ_0 I_{\text{innen}}$ \\
Spule: $B = μ_0 n I$ \\
Biot-Savart-Gesetz: \\
$\displayed{\v B(\v r) = \frac{μ_0}{4π} ∫ \frac{\v j(\v r') × (\v r - \v r')}{\abs{\v r - \v r'}^3} \d V'}$ \\
$\displayed{\v B(\v r) = \frac{μ_0 I}{4 π} ∫ \frac{\d \v s' × (\v r - \v r ')}{\abs{\v r - \v r'}^3}}$ \\
Z-Feld einer Leiterschleife: \\
$\displayed{B_z = \frac{μ_0 I}{4π} \frac{2π R^2}{(z^2 + R^2)^{3/2}}}$ \\
Leiterschleife ($r \gg R$): \\
$\displayed{\v B(\v r) = \frac{μ_0}{4π}(3 \frac{\v μ · \v r}{r^5} \v r - \frac{1}{r^5} \v μ)}$ \\
Magnetisierung: $\v M = \frac{1}{V} \sum_i \v μ_i$ \\
Magnetfeld aufgrund der Magnetisierung $M$: \\
$\v B_{mag} = μ_0 \frac{I_m}{l} \hat n = μ_0 \v M$ \\
$\v B = \v B_0 + \v B_{mag} = \v B_0 + μ_0 \v M$ \\
Magnetische Erregung: $\v H := \frac{1}{μ_0} \v B - \v M$ \\
$\v B = μ_0(\v H + \v M)$ \\
$∮ \v H \d \v s = I_{\text{frei}}$ \\
Magnetische Suszeptibililät: $χ_m$ \\
Magnetisierung: $\v M = χ_m \v H$ \\
$μ_0 \v H = \v B - μ_0 \v M$ \\
Relative Permeabilität: $μ = χ_m + 1$ \\
$\v B = μ_0 μ \v H$ \\
$χ_m > 0, μ > 1$: Paramagnetismus \\
$χ_m < 0, μ < 1$: Diamagnetismus \\
$χ_m \gg 0, μ \gg 1$: Ferromagnetismus \\
Sättigungsmagnetisierung: $\v M_s$ \\
Curie-Gesetz: $\displayed{\v M = \frac{1}{3} \frac{μ B_{ext}}{k_B T} \v M_s \sim \frac{1}{T}}$ \\
Grenzflächen mit unterschiedlichen $μ$: \\
\[\begin{array}{ll}H_{\parallel}^{(1)} = H_{\parallel}^{2} & \frac{B^{(1)}_{\parallel}}{μ_1} = \frac{B^{(2)}_{\parallel}}{μ_2} \\ B_{\perp}^{(1)} = B_{\perp}^{2} & μ_1 H^{(1)}_{\perp} = μ_2 H^{(2)}_{\perp} \end{array}\]
\section{Induktion}
Farardaysches Induktionsgesetz: \\
$\displayed{U_{ind} = ∮ \v E_{ind} \d \v s = - \dot ϕ_m}$ \\
Leiterschleife: $U_{ind} = v B l$ \\
Induktionsgesetz: $\displayed{∮ \v E \d \v s = - \dd{}{t} ∫ \v B \d \v A}$ \\
Lenzsche Regel: Induktion wirkt der Ursache stets entgegen. \\
Rotierende Leiterschleife: \\
$\displayed{ϕ_m = ∫ \v B \d \v A = B A \cos{ω t}}$ \\
$U_{ind} = - \dot ϕ_m(t) = ω B A \sin{ω t}$ \\
Induktivität $L$: $ϕ_m = L · I$ \\
Induktivitäten: \\
Spule: $L = μ μ_0 \frac{N^2}{l} A$ \\
Drahtschleife: $L = μ_0 R \ln \frac{R}{r}$ \\
Doppelleitung: $L = \frac{μ_0 l}{π} \ln \frac{a}{r}$ \\
Koaxialkabel: $L = \frac{μ_0 l}{2 π} \ln \frac{r_a}{r_i}$ \\
Selbstinduktion: $U_{ind} = - ϕ_m = - L \dot I$ \\
Ampere-Maxwell-Gesetz: \\
$\displayed{∮ \v B \d \v s = μ_0 ∫ \v j \d \v A + μ_0 ε_0 ∫ \pp{\v E}{t} \d \v A}$ \\
In Materie: $\displayed{∮ \v H \d \v s = μ_0 ∫ \v j \d \v A + μ_0 ε_0 ∫ \pp{\v D}{t} \d \v A}$
\section{Schaltungen}
\subsection{Gleichstrom}
LR-Glied: \\
Einschalten: $\displayed{I(t) = \frac{U_0}{R}(1 - e^{-\frac{tR}{L}})}$ \\
Auschalten: $\displayed{I(t) = \frac{U_0}{R} e^{-\frac{tR}{L}}}$ \\
RC-Glied: \\
Einschalten: $\displayed{I(t) = \frac{U_0}{R} e^{-\frac{t}{RC}}}$ \\
Ausschalten: $\displayed{I(t) = -\frac{U_0}{R} e^{-\frac{t}{RC}}}$ \\
\subsection{Wechselstrom}
Wirkleistung: \\
$P = U_{\text{eff}} I_{\text{eff}} = \frac{1}{2} U_0 = \frac{1}{T} ∫_0^T P(t) \d t$ \\
Blindleistung: $Q$ \\
Scheinleistung: $S^2 = P^2 + Q^2$ \\
Induktiver Widerstand: \\
$U(t) = U_0 \cos ω t$ \\
$I(t) = I_0 \cos(ω t - \frac{π}{2})$ \\
$I_0 = \frac{U_0}{ω L}$ \\
Kapazitiver Widerstand: \\
$U(t) = U_0 \cos ω t$ \\
$I(t) = I_0 \cos(ω t + \frac{π}{2})$ \\
$I_0 = ω C U_0$ \\
Ohmscher Widerstand: $Z_R = R, φ = \SI{0}{\degree}$ \\
Induktiver Widerstand: $Z_L = ω L, φ = \SI{-90}{\degree}$ \\
Kapazitiver Widerstand: $Z_C = \frac{1}{ω C}, φ = \SI{90}{\degree}$
\subsection{Komplexe Darstellung}
$U(t) = U_0 e^{i ω t}$ \\
$I(t) = I_0 e^{i( ω t + φ)} = I_0 e^{i φ} e^{i ω t}$ \\
Ohmscher Widerstand: \\
$U(t) = U_0 e^{i ω t}$ \\
$I(t) = \frac{U_0}{R} e^{i ω t} = \hat I e^{i ω t}$ \\
Induktiver Widerstand: \\
$U(t) = U_0 e^{i ω t}$ \\
$I(t) = \frac{U_0}{i ω L} e^{i ω t} = \hat I e^{i ω t}$ \\
Kapazitiver Widerstand: \\
$U(t) = U_0 e^{i ω t}$ \\
$I(t) = i ω C U_0 e^{i ω t} = \hat I e^{i ω t}$ \\
Impedanzen: \\
$\hat Z_R = R$ \\
$\hat Z_L = i ω L = ω L e^{i \frac{π}{2}}$ \\
$\hat Z_C = \frac{1}{i ω C} = \frac{1}{ω C} e^{- i \frac{π}{2}}$ \\
Ohmsches Gesetz: $\hat U = \hat Z · \hat I$ \\
RC-Serienschaltung: \\
$I(t) = I_0 e^{i ω t}$ \\
$U(t) = \hat Z I_0 e^{i ω t}$ \\
$\hat Z = (R + \frac{1}{i ω C})$ \\
RC-Parallelschaltung: \\
$U(t) = U_0 e^{i ω t}$ \\
$I(t) = U_0 \frac{1}{\hat Z} e^{i ω t}$ \\
$\frac{1}{\hat Z} = \frac{1}{R} + i ω C$ \\
Erweiterte Kirchhoffsche Regeln: \\
Knotenregel: $\sum \hat I = 0$ \\
Maschenregel: $\sum \hat U = 0$ \\
Reihenschaltung: $\hat Z = \sum \hat Z_i$ \\
Parallelschaltung: $\hat Z^{-1} = \sum \hat Z_i^{-1}$ \\
RLC-Schwingkreis ohne Stromquelle: \\
$\displayed{I(t) = C_1 e^{- γ t} e^{i ω_R t} + C_2 e^{-γt} e^{-i  ω_R t}}$ \\
$ω_R = (ω_0^2 - γ^2), ω_0^2 = \frac{1}{LC}, γ = \frac{R}{2L}$ \\
$γ < ω_0$: Schwingfall \\
$γ > ω_0$: Kriechfall \\
$γ = ω_0$: aperiodischer Grenzfall \\
Mit Stromquelle: \\
$\displayed{I(t) = U_0 \frac{1}{\hat Z} e^{i ω t} = e^{i ω t}·}$ \\
$\displayed{(\frac{U_0 R}{R^2 + (ω L - \frac{1}{ωC})^2} - i \frac{U_0(ω L - \frac{1}{ω C})}{R^2 + (ω L - \frac{1}{ω C})^2})}$ \\
Unbelasteter (verlustfreier) Transformator: \\
$\displayed{U_2 = \frac{N_2}{N_1} U_1}$ \\
$\displayed{I_2 = \frac{N_1}{N_2} I_1}$ \\
Elektrische Leistung im RC-Glied / elektrische Feldenergie: $W_{\text{el}} = \frac{1}{2} C U(t)^2$ \\
Elektrische Leistung im LR-Glied / magnetische Feldenergie: $W_{\text{el}} = \frac{1}{2} L I(t)^2$ \\
Energiedichte des elektrischen Feldes: \\
$ω_{\text{el}} = \frac{1}{2} ε ε_0 E^2 = \frac{1}{2} E D$ \\
Energiedichte des magnetischen Feldes: \\
$ω_{\text{m}} = \frac{1}{2} μ μ_0 H^2 = \frac{1}{2} B H$ \\
Allgemein: $ ω_{\text{elektromag}} = \frac{1}{2}(\v E · \v E + \v B · \v H)$
\section{Elektromagnetische  Wellen}
Transversale Wellen \\
$\to$ Ausbreitung senkrecht zur Schwingungsrichtung \\
Longitudianle Wellen \\
$\to$ Ausbreitung entlang der Schwhingungsrichtung \\
Harmonische Ebene Welle: \\
$y(x, t) = A \sin(k x \pm ω t)$ \\
$y(\v x, t) = A \sin(\v k \v x \pm ω t)$ \\
Wellenzahl: $k = \frac{2π}{λ}$ \\
Wellenlänge: $λ = \frac{2π}{k}$ \\
Phasengeschwindigkeit: $v_{ph} = \frac{ω}{k}$ \\
Amplitude $A$ \\
Winkelgeschwindigkeit: $ω = 2 π v = \frac{2 π}{T}$ \\
Fourier-Reihe: \\
$\displayed{f(t) = f(t + T)}$ \\
$\displayed{\to f(t) = \frac{a_0}{2} + \sum_{n = 1}^{∞} a_n \cos(n ω t) + b_n \sin(n ω t)}$ \\
$\displayed{a_i = \frac{2}{T} ∫_{-T/2}^{T/2} f(t) \cos(i ω t) \d t}$ \\
$\displayed{b_i = \frac{2}{T} ∫_{-T/2}^{T/2} f(t) \sin(i ω t) \d t}$ \\
Fourier-Integral: \\
$\displayed{f(t) = \frac{1}{π} ∫_0^∞(a(ω) \cos(ω t) + b(ω) \sin(ω t)) \d ω}$ \\
$\displayed{a(ω) = ∫_{-∞}^∞ f(t) \cos(ω t) \d t}$ \\
$\displayed{b(ω) = -∫_{-∞}^∞ f(t) \sin(ω t) \d t}$ \\
Klassische Wellengleichung: \\
$\displayed{\frac{\partial^2 ψ}{\partial t^2} = v^2 \frac{\partial^2 ψ}{\partial x^2}}$ \\
$\displayed{\frac{\partial^2 ψ}{\partial t^2} = v^2 Δ ψ}$ \\
Wellenpaket: \\
$\displayed{ψ(x, t) = \frac{1}{\sqrt{2π}} ∫_{-∞}^{+∞} A(k) e^{i(k x - ω t)} \d k}$ \\
$\displayed{A(k) = \frac{1}{\sqrt{2π}} ∫_{-∞}^{+∞} ψ(x, 0) e^{-i(kx)}} \d x$ \\
Phasengeschwindigkeit: $v_{ph} = \frac{ω}{k}$ \\
Dispersionsrelation: \\
$\displayed{ω = ω(k) = ω(k_0) + \dd{ω}{k}|_{k_0} (k - k_0) + \dots}$ \\
Gruppengeschwindigkeit: $v_{\text{gr}} = \dd{ω}{k}$ \\
Elektromagnetische Wellengleichungen:
\[\begin{array}{cc}
    \frac{\partial^2 \v E}{\partial t^2} = \frac{1}{ε_0 μ_0} Δ \v E & \frac{\partial^2 \v E}{\partial t^2} = \frac{1}{ε μ ε_0 μ_0} Δ \v E \\
    \frac{\partial^2 \v B}{\partial t^2} = \frac{1}{ε_0 μ_0} Δ \v B & \frac{\partial^2 \v B}{\partial t^2} = \frac{1}{ε μ ε_0 μ_0} Δ \v B
  \end{array}\]
Linear polarisierte, ebene Wellen:
$\v E(\v x, t) = \hat e_y E_0 \sin(k x - ω t)$ \\
$\v B(\v x, t) = \hat e_z \frac{E_0}{c} \sin(k x - ω t)$ \\
$\v E \perp \v B, \v E, \v B \perp \v k$ \\
$\v E, \v B$ in Phase \\
$\abs{\v B} = \frac{\abs{\v E}}{c}$ \\
$\v B = \frac{1}{ω}(\v k × \v E)$ \\
$c = \frac{ω}{k}$ \\
Zirkular polarisierte, ebene Wellen: \\
$E_{0,x} = E_{0, y}; φ = \SI{90}{\degree}$ \\
Elliptisch polarisierte, ebene Wellen: \\
$E_{0, x} \neq E_{0, y}; φ = \SI{90}{\degree}$ \\
Kugelwellen: $\v E = \frac{\v E_0}{r} \sin(\v k \v r - ω t), \v B = \frac{\v B_0}{r} \sin(\v k \v r - ω t), \v k \parallel \v r$ \\
Poynting-Vektor: $\v S = \frac{1}{μ_0}(\v E × \v B) = \v E × \v H$ \\
Hertzer Dipol: \\
Eigenfrequenz: $\frac{1}{\sqrt{L C}}$ \\
Nahfeld: $E \sim \frac{1}{r^3}, B \sim \frac{1}{r^2}, φ = \SI{90}{\degree}$ \\
Fernfeld: $E \sim \frac{1}{r}, B \sim \frac{1}{r}, \v E \perp \v B \perp \v k$ \\
Drahtwelle: \\
$I_0(z) = I_0 \cos(\frac{π x}{e})$ \\
$U_0(z) = I_0 \sin(\frac{π x}{e})$ \\
Strahlungsgleichung: \\
$\displayed{S(r, v) = \frac{p_0^2 ω^4 \sin^2 θ}{16 π^2 e_0 c^3 r^2} \sin^2(ω t - k r)}$ \\
$λ = c / ν$
\section{Optik}
Phasengeschwindigkeit: \\
$\displayed{c = \frac{1}{\sqrt{ε_0 μ_0}}, c' = \frac{1}{\sqrt{ε μ ε_0 μ_0}}}$ \\
Amplitude: $E = c B$ \\
Energiefluss: $\v S = \frac{1}{μ_0}(\v E × \v B)$ \\
Intensität: $I = \abs{\v S} = ε_0 c E^2$ \\
Spalt, Nullstellen: $a \sin θ = m λ, m ∈ ℕ$ \\
Intensitätsverteilung für Beugung am Einzelspalt: \\
$\displayed{I = I_0 (\frac{\sin(\frac{π a}{λ} \sin θ)}{\frac{π a}{λ} \sin θ})^2}$ \\
Doppelspalt mit unendlich dünne Spaltbreite: \\
Interferenzmaxima: $d \sin θ = m λ, m ∈ ℕ_0$ \\
Interferenzminima: $d \sin θ = (m + \frac{1}{2}) λ, m ∈ ℕ_0$ \\
Intensitätsverteilung: $I = 4 I_0 \cos^2 \frac{1}{2} δ$ \\
$δ = \frac{2π}{λ} d \sin θ$ \\
Doppelspalt mit Spaltbreite $a$: \\
$\displayed{I = 4 I_0 (\frac{\sin \frac{1}{2} ϕ}{\frac{1}{2} ϕ})^2 \cos^2 \frac{1}{2} δ}$ \\
$\displayed{ϕ = \frac{2 π}{λ} a \sin θ}$ \\
Fraunhofer-Beugung: \\
$\displayed{E(θ, t) \sim e^{i ω t} ∫ A(x) e^{i K x} \d x, K = k \sin θ}$ \\
Mehrfachspalt: \\
$\displayed{I(θ) \sim (\frac{\sin(M k \frac{d}{2})}{\sin(k \frac{d}{2})})^2 (\frac{\sin(k \frac{a}{2})}{k \frac{a}{2}})^2}$ \\
Hauptmaxima: $\sin ϕ_n = n \frac{λ}{d}, m ∈ ℕ_0$ \\
Auflösungsvermögen: $\frac{λ}{Δ λ} \leq n m$ \\
Anzahl Spalte: $m$ \\
Ordnung: $n$ \\
Gesetz von Snellius: $n_1 \sin θ_1 = n_2 \sin θ_2$ \\
Kritischer Winkel: $\sin θ_k = \frac{n_1}{n_2}$ \\
Gesetz von Malus: $I = I_0 \cos^2 θ$ \\
Fresnelsche Formeln: \\
$\displayed{R_{\perp}(α, β) = (\frac{\sin(α - β)}{\sin(α + β)})^2}$ \\
$\displayed{R_{\parallel}(α, β) = (\frac{\tan(α - β)}{\tan(α + β)})^2}$ \\
$\displayed{T_{\perp}(α, β) = (\frac{2 \sin β \cos α}{\sin(α + β)})^2}$ \\
$\displayed{T_{\parallel}(α, β) = (\frac{2 \sin β \cos α}{\sin(α + β)})^2}$ \\
Gesetz von Brewster: $\tan θ_{Br} = \frac{n_2}{n_1}$ \\
Reflexionskoeffizient bei senkrechten Einfall: \\
$\displayed{R_{\parallel} = R_{\perp} = (\frac{n_2 - n_1}{n_2 + n_1})^2}$ \\
Linsengleichung: $\frac{1}{g} + \frac{1}{b} = \frac{1}{f}$ \\
Brennweite: $\displayed{f = \frac{r_1 r_2}{(n - 1)(r_1 + r_2)}}$ \\
$\frac{B}{G} = \frac{b}{g}$ \\
Parallelstrahl $\to$ Brennstrahl \\
Zentralstrahl $\to$ Zentralstrahl \\
Brennstrahl $\to$ Parallelstrahl
\section{Werte}
$ε_0 = \SI{8.85416e-12}{\coulomb\squared\per{\newton\meter\squared}}$ \\
$μ_0 = \SI{1.2566e-6}{\volt\second\per{\ampere\meter}}$ \\
$h = \SI{6.626e-34}{\joule\second}$ \\
$e = \SI{1.602}{\coulomb}$
\section{Extras}
$μ_0 ε_0 = \frac{1}{c^2}$ \\
$\hat Z = \abs{\hat Z} · e^{i φ}$
$\abs{\hat Z} = \sqrt{(\Re \hat Z)^2 + (\Im \hat Z)^2}$ \\
$φ = \arctan{\frac{\Im \hat Z}{\Re \hat Z}}$
Maxwell-Gleichungen: \\
$\displayed{\v ∇ · \v E = \frac{ρ}{ε_0}}$ \\
$\displayed{\v ∇ · \v B = 0}$ \\
$\displayed{\v ∇ × \v E = - \pp{\v B}{t}}$ \\
$\displayed{\v ∇ × \v B = μ_0 \v j + μ_0 ε_0 \pp{\v E}{t}}$ \\
Makroskopische: \\
$\displayed{\v ∇ · \v D = ρ_{\text{frei}}}$ \\
$\displayed{\v ∇ · \v B = 0}$ \\
$\displayed{\v ∇ × \v E = - \pp{\v B}{t}}$ \\
$\displayed{\v ∇ × \v H = \v j + \pp{\v D}{t}}$ \\
$ρ_{frei} = ρ - ρ_{pol} = ρ + \div \v P$ \\
Integrale Form: \\
$\displayed{∮_{\partial V} \v D \d \v A = ∫_V ρ \d V = Q(V)}$ \\
$\displayed{∮_{\partial V} \v B \d \v A = 0}$ \\
$\displayed{U_{\text{ind}} = ∮_{\partial A} \v E \d \v s = - ∫_A \pp{\v B}{t} \d \v A}$ \\
$\displayed{∮_{\partial A} \v H \d \v s = ∫_A \v j \d \v A + ∫_A \pp{\v D}{t} \d \v A}$ \\
Tiefpass: \\
$\hat U_e = \hat U_R + \hat U_C, \hat U_C = \hat U_a, \hat I_R = \hat I_C = \hat I_0$ \\
$\to \hat U_e = \hat Z_R \hat I_0 + \hat Z_C \hat I_0$ \\
$\to \hat I_0 = \frac{\hat U_e}{\hat Z_R + \hat Z_C}$ \\
$\to \hat U_C = \hat U_a = \hat U_e \frac{\hat Z_C}{\hat Z_R + \hat Z_C}$ \\
Ringspule mit Lücke: \\
$μ_1 H_{\perp}^{(1)} = μ_2 H_{\perp}^{(2)} \to μ_{\text{Fe}} H_{\text{Fe}} = H_{\text{Luft}}$ \\
$∮ \v H \d \v s = I_i = N I$ \\
$∮ \v H \d \v s = ∫_{\text{Luft}} H_{\text{Luft}} \d \v s + ∫_{\text{Fe}} H_{\text{Fe}} \d \v s$ \\
$∮ \v H \d \v s = ∫_{\text{Luft}} H_{\text{Luft}} \d \v s + ∫_{\text{Fe}} \frac{H_{\text{Luft}}}{μ_{\text{Fe}}} \d \v s$ \\
$N I = H_{\text{Luft}} d + \frac{H_{\text{Luft}}}{μ_{\text{Fe}}}(2 π R - d)$ \\
$\displayed{\to H_{\text{Luft}} = \frac{μ_{\text{Fe}} N I}{2 π R + (μ_{\text{Fe}} - 1) d}}$ \\
$B_{\text{Luft}} = μ_0 H_{\text{Luft}} = B_{\text{Fe}}$ \\
Ring mit Paramagnet: \\
$B_{\text{Fe}} = μ_0 μ_{\text{Fe}} H_{\text{Fe}} = B_{\text{Para}}$ \\
$H_{\text{Para}} = \frac{B_{\text{Para}}}{μ_0} - M$
$∮ \v H \d \v s = 0 = ∫_{\text{Fe}} H_{\text{Fe}} \d \v s + ∫_{\text{Para}} H_{\text{Para}} \d \v s$ \\
$0 = H_{\text{Fe}}(2 π R - d) + (\frac{B_{\text{Para}}}{μ_0} - M) d$ \\
$0 = H_{\text{Fe}}(2 π R - d) + (μ_{\text{Fe}} H_{\text{Fe}} - M) d$ \\
$\displayed{\to H_{\text{Fe}} = \frac{d M}{2 π R + (μ_{\text{Fe}} - 1) d}}$
\end{multicols*}
\end{document}
